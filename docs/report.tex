\documentclass{article} % Especially this!

%%%%%%%%%%%%%%%%%%%%%%%%%%%%%%%%%%%%%%%%%%%%%%%%%%%%%%%%

\usepackage[english]{babel}
\usepackage[utf8]{inputenc}
\usepackage[a4paper, total={6in, 8.5in}]{geometry}
\usepackage{amsmath}
\usepackage{amsthm}
\usepackage{amsfonts}
\usepackage{amssymb}
\usepackage[usenames,dvipsnames]{xcolor}
\usepackage{graphicx}
%\usepackage[siunitx]{circuitikz}
\usepackage{tikz}
\usepackage[colorinlistoftodos, color=orange!50]{todonotes}
\usepackage{hyperref}
\usepackage[numbers, square]{natbib}
\usepackage{fancybox}
\usepackage{epsfig}
\usepackage{soul}
\usepackage{listings}
\usepackage[framemethod=tikz]{mdframed}
\usepackage[shortlabels]{enumitem}
\usepackage[version=4]{mhchem}
\usepackage{multicol}
\usepackage{pgfplots}
\usepackage{version}
\pgfplotsset{compat=1.14}
\usepackage{graphicx}
\usepackage{subfig}

%%%%%%%%%%%%%%%%%%%%%%%%%%%%%%%%%%%%%%%%%%%%%%%%%%%%%%%

% SYNTAX FOR NEW COMMANDS:
%\newcommand{\new}{Old command or text}

% EXAMPLE:

\newcommand{\blah}{blah blah blah \dots}

\setlength{\marginparwidth}{3.4cm}


% NEW COUNTERS
\newcounter{points}
\setcounter{points}{100}
\newcounter{spelling}
\newcounter{english}
\newcounter{units}
\newcounter{other}
\newcounter{source}
\newcounter{concept}
\newcounter{missing}
\newcounter{math}
\newcounter{terms}
\newcounter{clarity}
\newcounter{late}

% COMMANDS
\newtheorem{theorem}{Theorem}

\newcommand{\late}{\todo{late submittal (-5)}
\addtocounter{late}{-5}
\addtocounter{points}{-5}}

\definecolor{pink}{RGB}{255,182,193}
\newcommand{\hlp}[2][pink]{ {\sethlcolor{#1} \hl{#2}} }

\definecolor{myblue}{rgb}{0.668, 0.805, 0.929}
\newcommand{\hlb}[2][myblue]{ {\sethlcolor{#1} \hl{#2}} }

\newcommand{\clarity}[2]{\todo[color=CornflowerBlue!50]{CLARITY of WRITING(#1) #2}\addtocounter{points}{#1}
\addtocounter{clarity}{#1}}

\newcommand{\other}[2]{\todo{OTHER(#1) #2} \addtocounter{points}{#1} \addtocounter{other}{#1}}

\newcommand{\spelling}{\todo[color=CornflowerBlue!50]{SPELLING (-1)} \addtocounter{points}{-1}
\addtocounter{spelling}{-1}}
\newcommand{\units}{\todo{UNITS (-1)} \addtocounter{points}{-1}
\addtocounter{units}{-1}}

\newcommand{\english}{\todo[color=CornflowerBlue!50]{SYNTAX and GRAMMAR (-1)} \addtocounter{points}{-1}
\addtocounter{english}{-1}}

\newcommand{\source}{\todo{SOURCE(S) (-2)} \addtocounter{points}{-2}
\addtocounter{source}{-2}}
\newcommand{\concept}{\todo{CONCEPT (-2)} \addtocounter{points}{-2}
\addtocounter{concept}{-2}}

\newcommand{\missing}[2]{\todo{MISSING CONTENT (#1) #2} \addtocounter{points}{#1}
\addtocounter{missing}{#1}}

\newcommand{\maths}{\todo{MATH (-1)} \addtocounter{points}{-1}
\addtocounter{math}{-1}}
\newcommand{\terms}{\todo[color=CornflowerBlue!50]{SCIENCE TERMS (-1)} \addtocounter{points}{-1}
\addtocounter{terms}{-1}}


\newcommand{\summary}[1]{
\begin{mdframed}[nobreak=true]
\begin{minipage}{\textwidth}
\vspace{0.5cm}
\begin{center}
\Large{Grade Summary} \hrule 
\end{center} \vspace{0.5cm}
General Comments: #1

\vspace{0.5cm}
Possible Points \dotfill 100 \\
Points Lost (Late Submittal) \dotfill \thelate \\
Points Lost (Science Terms) \dotfill \theterms \\
Points Lost (Syntax and Grammar) \dotfill \theenglish \\
Points Lost (Spelling) \dotfill \thespelling \\
Points Lost (Units) \dotfill \theunits \\
Points Lost (Math) \dotfill \themath \\
Points Lost (Sources) \dotfill \thesource \\
Points Lost (Concept) \dotfill \theconcept \\
Points Lost (Missing Content) \dotfill \themissing \\
Points Lost (Clarity of Writing) \dotfill \theclarity \\
Other \dotfill \theother \\[0.5cm]
\begin{center}
\large{\textbf{Grade:} \fbox{\thepoints}}
\end{center}
\end{minipage}
\end{mdframed}}

%#########################################################

%To use symbols for footnotes
\renewcommand*{\thefootnote}{\fnsymbol{footnote}}
%To change footnotes back to numbers uncomment the following line
%\renewcommand*{\thefootnote}{\arabic{footnote}}

% Enable this command to adjust line spacing for inline math equations.
% \everymath{\displaystyle}

%%%%%%%%%%%%%%%%%%%%%%%%%%%%%%%%%%%%%%%

\title{
\normalfont \normalsize 
\textsc{Pattern Recognition, 2021} \\ 
[10pt] 
\rule{\linewidth}{0.5pt} \\[6pt] 
\huge 
TODO
\rule{\linewidth}{2pt}  \\[10pt]
}
\author{Lorenzo Sani}
\date{\normalsize \today}

\begin{document}

\maketitle

\tableofcontents

%###############################################
\section{Abstract}

%###############################################
\section{Introduction}
Federated Learnign (FL) is a newly introduced approach to collaborative machine 
learning. There are nowadays many traditional machine learning algorithms which 
require huge quantities of data raining examples to learn. The rising problem is 
that is often very difficult to collect a sufficient amount of data to reach a 
reasonable reliability. This problem is very common in settings in which data 
access is restricted by righteous privacy regulations, e.g. personal healthcare, 
or customization of personal devices. The implementation of a FL setting allows 
to avoid the necessity to collect data in a single place. 

A typical FL setting 
is built by taking into account a set of clients and a centralized server. The 
main steps of a FL iterative algorithm are the following:
\begin{enumerate}
    \item global model initialization;
    \item global model distribution;
    \item local models training;
    \item local models aggregation;
    \item repeat from 2.
\end{enumerate}
Firstly, step (1), the server, usually, initialize a unique global model to be 
passed to the clients and keep the consistency of the procedure, i.e. ensuring 
that every client runs the same learning algorithm.
Then, step (2), the server is in charge to distribute the initialized model to 
all the available clients. FL algorithms allow to check how many clients are 
available and train the same even if they are a low number. The distribution 
procedure could take into account the numerosity of the single clients' datasets 
(the only one information on the dataset coming from the clients).
After having received the global model, all the available clients train locally
the global model with their data, step (3), producing a new local model.
The weights of all the local models are then aggregated, step (4), at the server 
place to be reduced to a new global model. The aggregation procedure, as the 
distribution prevoiusly, could take into account the numerosity of the clients' 
datasets.
Moreover the evaluation step, always performed locally by the clients, could be
aggregated to the server in an analogous procedure than local models. This allows
the server to have global perception of the performance of the learning step by 
step. This action usually is inserted between step (2) and step (3).
The iterative procedure is finally repeated until the number of fixed federated 
iterations is reached, usually.

Before presnting the methods used, it is necessary to spend some words on the 
aggregation method used. Despite FL is a newly introduced approch, many 
aggregation procedures have been proposed. It is important to say that FL is 
not widely understood by now and because of this not any aggregation procedure, 
or more generally FL algorithm, is reliable for any specific problem. The 
aggregation method used in this analysis is, probably, the most general: 
\verb|FedAvg|.

%###############################################
\section {Methods}

%###############################################
\section {Results}

%###############################################
\section {Conclusions}

%###############################################
\section {Figures}

%###############################################

\section{Bibliography}
\begin{thebibliography}{90}
\bibitem{CuckerSmale}F. Cucker, S. Smale. "On the mathematics of emergence", The Mathematical Society of Japan and Springer 2007, Published online: 28 March 2007.
\bibitem{CuckerSmale1}F. Cucker, S. Smale. "Emergent Behaviour in Flocks", IEEE Transaction on Automatic Control, Vol. 52, No. 5, May 2007.
\bibitem{repo}\verb|https://github.com/relogu/PofCS_project|.
\end{thebibliography}

\newpage
\appendix
\renewcommand\thefigure{\thesection.\arabic{figure}}
\end{document} % NOTHING AFTER THIS LINE IS PART OF THE DOCUMENT