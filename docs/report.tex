\documentclass{article} % Especially this!

%%%%%%%%%%%%%%%%%%%%%%%%%%%%%%%%%%%%%%%%%%%%%%%%%%%%%%%%

\usepackage[english]{babel}
\usepackage[utf8]{inputenc}
\usepackage[a4paper, total={6in, 8.5in}]{geometry}
\usepackage{amsmath}
\usepackage{amsthm}
\usepackage{amsfonts}
\usepackage{amssymb}
\usepackage[usenames,dvipsnames]{xcolor}
\usepackage{graphicx}
%\usepackage[siunitx]{circuitikz}
\usepackage{tikz}
\usepackage[ruled,vlined]{algorithm2e}
\usepackage[colorinlistoftodos, color=orange!50]{todonotes}
\usepackage{hyperref}
\usepackage[numbers, square]{natbib}
\usepackage{fancybox}
\usepackage{epsfig}
\usepackage{soul}
\usepackage{listings}
\usepackage[framemethod=tikz]{mdframed}
\usepackage[shortlabels]{enumitem}
\usepackage[version=4]{mhchem}
\usepackage{multicol}
\usepackage{pgfplots}
\usepackage{version}
\pgfplotsset{compat=1.14}
\usepackage{graphicx}
\usepackage{subfig}

%%%%%%%%%%%%%%%%%%%%%%%%%%%%%%%%%%%%%%%%%%%%%%%%%%%%%%%

% SYNTAX FOR NEW COMMANDS:
%\newcommand{\new}{Old command or text}

% EXAMPLE:

\newcommand{\blah}{blah blah blah \dots}

\setlength{\marginparwidth}{3.4cm}


% NEW COUNTERS
\newcounter{points}
\setcounter{points}{100}
\newcounter{spelling}
\newcounter{english}
\newcounter{units}
\newcounter{other}
\newcounter{source}
\newcounter{concept}
\newcounter{missing}
\newcounter{math}
\newcounter{terms}
\newcounter{clarity}
\newcounter{late}

% COMMANDS
\newtheorem{theorem}{Theorem}

\newcommand{\late}{\todo{late submittal (-5)}
\addtocounter{late}{-5}
\addtocounter{points}{-5}}

\definecolor{pink}{RGB}{255,182,193}
\newcommand{\hlp}[2][pink]{ {\sethlcolor{#1} \hl{#2}} }

\definecolor{myblue}{rgb}{0.668, 0.805, 0.929}
\newcommand{\hlb}[2][myblue]{ {\sethlcolor{#1} \hl{#2}} }

\newcommand{\clarity}[2]{\todo[color=CornflowerBlue!50]{CLARITY of WRITING(#1) #2}\addtocounter{points}{#1}
\addtocounter{clarity}{#1}}

\newcommand{\other}[2]{\todo{OTHER(#1) #2} \addtocounter{points}{#1} \addtocounter{other}{#1}}

\newcommand{\spelling}{\todo[color=CornflowerBlue!50]{SPELLING (-1)} \addtocounter{points}{-1}
\addtocounter{spelling}{-1}}
\newcommand{\units}{\todo{UNITS (-1)} \addtocounter{points}{-1}
\addtocounter{units}{-1}}

\newcommand{\english}{\todo[color=CornflowerBlue!50]{SYNTAX and GRAMMAR (-1)} \addtocounter{points}{-1}
\addtocounter{english}{-1}}

\newcommand{\source}{\todo{SOURCE(S) (-2)} \addtocounter{points}{-2}
\addtocounter{source}{-2}}
\newcommand{\concept}{\todo{CONCEPT (-2)} \addtocounter{points}{-2}
\addtocounter{concept}{-2}}

\newcommand{\missing}[2]{\todo{MISSING CONTENT (#1) #2} \addtocounter{points}{#1}
\addtocounter{missing}{#1}}

\newcommand{\maths}{\todo{MATH (-1)} \addtocounter{points}{-1}
\addtocounter{math}{-1}}
\newcommand{\terms}{\todo[color=CornflowerBlue!50]{SCIENCE TERMS (-1)} \addtocounter{points}{-1}
\addtocounter{terms}{-1}}


\newcommand{\summary}[1]{
\begin{mdframed}[nobreak=true]
\begin{minipage}{\textwidth}
\vspace{0.5cm}
\begin{center}
\Large{Grade Summary} \hrule 
\end{center} \vspace{0.5cm}
General Comments: #1

\vspace{0.5cm}
Possible Points \dotfill 100 \\
Points Lost (Late Submittal) \dotfill \thelate \\
Points Lost (Science Terms) \dotfill \theterms \\
Points Lost (Syntax and Grammar) \dotfill \theenglish \\
Points Lost (Spelling) \dotfill \thespelling \\
Points Lost (Units) \dotfill \theunits \\
Points Lost (Math) \dotfill \themath \\
Points Lost (Sources) \dotfill \thesource \\
Points Lost (Concept) \dotfill \theconcept \\
Points Lost (Missing Content) \dotfill \themissing \\
Points Lost (Clarity of Writing) \dotfill \theclarity \\
Other \dotfill \theother \\[0.5cm]
\begin{center}
\large{\textbf{Grade:} \fbox{\thepoints}}
\end{center}
\end{minipage}
\end{mdframed}}

%#########################################################

%To use symbols for footnotes
\renewcommand*{\thefootnote}{\fnsymbol{footnote}}
%To change footnotes back to numbers uncomment the following line
%\renewcommand*{\thefootnote}{\arabic{footnote}}

% Enable this command to adjust line spacing for inline math equations.
% \everymath{\displaystyle}

%%%%%%%%%%%%%%%%%%%%%%%%%%%%%%%%%%%%%%%

\title{
\normalfont \normalsize 
\textsc{Pattern Recognition, 2021} \\ 
[10pt] 
\rule{\linewidth}{0.5pt} \\[6pt] 
\huge 
TODO
\rule{\linewidth}{2pt}  \\[10pt]
}
\author{Lorenzo Sani}
\date{\normalsize \today}

\begin{document}

\maketitle

\tableofcontents

%###############################################
\section{Abstract}

%###############################################
\section{Introduction}
Federated Learning (FL) is a newly introduced approach to collaborative machine 
learning \cite{9153560}. There are nowadays many traditional machine learning algorithms which 
require huge quantities of data raining examples to learn. The rising problem is 
that is often very difficult to collect a sufficient amount of data to reach a 
reasonable reliability. This problem is very common in settings in which data 
access is restricted by righteous privacy regulations, e.g. personal healthcare, 
or customization of personal devices. The implementation of a FL setting allows 
to avoid the necessity to collect data in a single place.

\subsection{Federated Learning}
A typical FL setting 
is built by taking into account a set of clients and a centralized server. The 
main steps of a FL iterative algorithm are the following:
\begin{enumerate}
    \item global model initialization;
    \item global model distribution;
    \item local models training;
    \item local models aggregation;
    \item repeat from 2.
\end{enumerate}
Firstly, step (1), the server, usually, initialize a unique global model to be 
passed to the clients and keep the consistency of the procedure, i.e. ensuring 
that every client runs the same learning algorithm.
Then, step (2), the server is in charge to distribute the initialized model to 
all the available clients. FL algorithms allow to check how many clients are 
available and train the same even if they are a low number. The distribution 
procedure could take into account the numerosity of the single clients' datasets 
(the only one information on the dataset coming from the clients).
After having received the global model, all the available clients train locally
the global model with their data, step (3), producing a new local model.
The weights of all the local models are then aggregated, step (4), at the server 
place to be reduced to a new global model. The aggregation procedure, as the 
distribution prevoiusly, could take into account the numerosity of the clients' 
datasets.
Moreover the evaluation step, always performed locally by the clients, could be
aggregated to the server in an analogous procedure than local models. This allows
the server to have global perception of the performance of the learning step by 
step. This action usually is inserted between step (2) and step (3).
The iterative procedure is finally repeated until the number of fixed federated 
iterations is reached, usually.

\subsection{Federated Average (FedAvg)}
Before presenting the methods used, it is necessary to spend some words on the 
aggregation method used. Despite FL is a newly introduced approch, many 
aggregation procedures have been proposed. It is important to say that FL is 
not widely understood by now and because of this not any aggregation procedure, 
or more generally FL algorithm, is reliable for any specific problem. The 
aggregation method used in this analysis is, probably, the most general: 
\verb|FedAvg|. This algorithm, firstly prosed in \cite{mcmahan2017communicationefficient}, relies on Stochastic 
Gradient Descent (SGD) optimization method, since the majority of the most succeful deep 
learning works were based on this. The available clients locally compute (step 3) their 
average gradient on their local data at the current model $w_t$, where $t$ 
identifies the federated round, and the central server aggregates these 
gradients and applies the update $w_{t+1}\leftarrow w_t - \eta\sum_{k=1}^K\frac{n_k}{n}g_k$.
Above, $g_k=\nabla F_k(w_t)$ is the average gradient of the client $k$, $\eta$ is 
the learning rate, $n_k$ is the number of samples at the client $k$, $n$ is the total 
number of samples (sum over all the available clients). Equivalently, the update 
can be given by $w_{t+1}\leftarrow w_t - \eta\sum_{k=1}^K\frac{n_k}{n}w_{t+1}^k$, where 
$w_{t+1}^k\leftarrow w_t - \eta g_k$ $\forall k$. In the last, every client takes 
a complete step of gradient descent, while the server only takes the weighted average 
of the resulting models. The following pseudo-code summarizes the procedure.
\begin{algorithm}[H]
    \SetAlgoLined
    \SetKwBlock{Ser}{Server executes:}{}
    \SetKwBlock{ClUp}{ClientUpdate($k$, $w_t$):}{}
    \SetKwFor{ParFor}{for}{do in parallel}{end}
    \Ser{
        initialize $w_0$\\
        \For(){each federated round $t=1,2,...$}{
            $S_t\leftarrow$ (select available clients)\\
            \ParFor{each client $k$ in $S_t$}{
                $w_{t+1}^k\leftarrow$ ClientUpdate($k$, $w_t$)
            }
            $w_{t+1}\leftarrow\sum_{k=1}^K\frac{n_k}{n}w_{t+1}^k$
        }
    }

    \ClUp{
        $\mathcal{B}\leftarrow$(split $k$-th client's dataset into batches of size $B$)\\
        \For(){each local epoch $i$ from $1$ to $E$}{
            \For{batch $b\in\mathcal{B}$}{
                $w\leftarrow w - \eta\nabla l(w;b)$
            }
        }
        return $w$ to server
    }
    \caption{$FedAvg$. The $K$ clients are indexed by $k$; $E$ is the number of local 
     epochs; $\eta$ is the learning rate; $B$ is the size of the local batches.}
\end{algorithm}

%###############################################
\section {Methods}

\subsection{Dataset}
A simple toy dataset was chosen to set up a classification toy model to perform some 
simulations in FL setting.
From the \verb|scikit-learn| Python package, which provides a wide set of generators 
for toy datsets, the \verb|datasets.make_moons| generator was picked up.
This function produces the requested number of points in a 2-D space drawing two 
interleaving circles, as Fig.\ref{} shows. 
The same function returns also the classification array, that relates every point to 
its corresponding circle.
One can make the request to add some noise to the generated points, and ask for the 
points to be shuffled, once generated.
The noise value was fixed to 0.1 along every simulation.
It is also possible to set the random state that seeds the noise and shuffling, if 
requested.
Two more functions, in addition to this settings, were built to trasform a little 
such generated dataset.
The first simply traslates the dataset by a given vector $(dx, dy)$, i.e. every 
point $(x, y)$ in the dataset undergoes the transformation $x'=x+dx$ and $y'=y+dy$.
The second applies a simple rotation by an angle $\theta$ with a standard transformation,
i.e. $x'=x\cos(\theta)-y\sin(\theta)$ and $y'=x\sin(\theta)-y\cos(\theta)$ following 
the above notation.
Examples of traslated and rotated datasets are shown in Fig.\ref{}.

\subsection{Model}
Every simulation was build on a simple two layers sequential model.
Both the layers are standard regular densely-connected Neural Network layers, the first 
with 4 output, the second with 2, since the model is expected to classify the points 
w.r.t. their circle of belonging.

\subsection{Simulation setting}
The framework adopted to make the necessary simulation is Flower: A Friendly Federated 
Learning Framework \cite{beutel2021flower}.
The federated framework is set up by running  a program for every client and one for 
the server.
These will communicate using a RPC (Remote Procedure Call) framework exchanging the 
weights of the model.
The ML framework used is \verb|tensorflow| with a \verb|keras.Sequential| model.
The client program accepts many parameters to build properly the client's dataset, and 
to set up the outputs also.
The server program has a minimal configuration, receiving the number of clients in the 
federated network and the number of the federated epochs to run.
Another program is used to simulate the same situation in a non-federated setting. 
In fact, since the seeds for the datasets' generations are fixed, it is able to build the 
equivalent aggregated dataset and train on a single, equivalently initialized, model.

%###############################################
\section {Results}
There were made many comparison between a FL setting and its equivalent aggregated one.
In every comparison the total number of samples was kept constant to 320, to be easily 
distributed over the three subdivision chosen (2, 4, and 8 clients).
The comparisons are intended to highlight the influence that the number of clients have 
on the performance of the learning model.
The expectation is that more the dataset is distributed lesser is the performance of 
the FL setting, i.e. more are the rounds needed to reach some values of accuracy and loss.
Whenever traslated or rotated datsets are used, the comparison is indended to extent 
studying the Trasfer Learning (TL) capacity of the FL setting, since every clients differs 
from the others also for the traslation vector $(dx, dy)$ or the rotation angle $\theta$.
There is no guess about the TL ability of these FL settings.
Every client's datset, as the aggregated one, are divided randomly in train set and test 
set usig the standard proportion 80-20, for train and test respectively.
In order to have consistent comparisons, the FL setting and the aggregated both are trained 
for a total of 2000 epochs.
The for every epoch an evaluation step is performed at every client's place as at the 
aggregated model.
The performance of the model at every step, represented by the loss and the accuracy, 
is retrieved and saved to be consulted later.
The loss function chosen is the Sparse Categorical Cross-Entropy \ref{?}, implemented by the
function \verb|tensorflow.keras.losses.SparseCategoricalCrossentropy|.
The accuracy is simply computed as the ratio between the number of well classified points 
and the total number of points in the test set.
The values of the losses and the accuracies are then plotted against the epoch at which they 
are computed.
Since the single client performance only does not give so much information about the 
performance of the whole FL model, the mean value of the losses and the mean value of the 
accuracies are then computed.
A confidence interval, that reach a confidence level of $95\%$, is then associated to 
these mean values, in order to give a more complete representation.
The distance between the curve of mean values of loss and accuracy between clients and the 
relative curve of the aggregated model gives a quantitative representation of far the two 
performances.
The wideness of the confidence intervals at the same time quantify the TL ability of the 
FL model.
%###############################################
\section {Conclusions}

%###############################################
\section {Figures}

%###############################################


\bibliographystyle{unsrt}
\bibliography{bibliography}

\newpage
\renewcommand\thefigure{\thesection.\arabic{figure}}
\end{document} % NOTHING AFTER THIS LINE IS PART OF THE DOCUMENT